\centerline{\Large{\textbf{Absztrakt}}}

\vspace{0.5cm}

\noindent Dolgozatom az egészségügyi adatvagyon hasznosításáról szól, melyben két különböző témára fókuszáltam: az egyik a kontrollcsoport-kiválasztás, a másik pedig információ kinyerése orvosi dokumentumokból. 

A megfigyeléses vizsgálatok gyakorta olyan eset-kontroll vizsgálatokon alapulnak, melyekben a következtetést (például egy gyógyszer hatását a gyógyulásra) az eset (kezelt) és a kontroll (kezeletlen) csoportok összehasonlítása alapján vonják le. Az eset-kontroll vizsgálatok megfelelő végrehajtásának alapvető kritériuma a megfelelő eset- és kontrollcsoportok kiválasztása. A retrospektív vizsgálatok során a kezelt csoport általában előre adott, és a vizsgálathoz olyan kontrollcsoportot kell kialakítani, amelyben az egyedek a vizsgált kérdést befolyásoló alaptulajdonságaik tekintetében nagy mértékben hasonlítanak az esetcsoport alanyaihoz. Dolgozatomban két új legközelebbi szomszéd alapú kontrollcsoport kiválasztási módszert javasoltam, amelyek az egyedek kiválasztását az eredeti n-dimenziós tu-lajdonságtérben végzik el. A javasolt módszerek hatékonyságát Monte Carlo szi-mulációkkal teszteltem az általam javasolt különbözőségi mérőszámok és más, szé-les körben használt hasonlósági mérőszámok felhasználásával. 

A mindennapi orvosi gyakorlatban a szívultrahang vizsgálatok eredményeit általában strukturálatlan szöveg formájában rögzítik, amelyekből a releváns információk kinyerése kihívásokkal teli feladat. Ezen információkinyerés támogatására kidolgoztam egy olyan szövegbányászaton alapuló információkinyerési módszert, amely a leletekben automatikusan azonosítja és egységesíti a szívultrahang mérések leírását, majd a kinyert és egységesített mérési leírásokat a mérési eredményekkel együtt strukturált formában tárolja. Nagy adathalmazon alapuló esettanulmányok révén kimutattam, hogy a javasolt módszerrel nagy biztonsággal nyerhetők ki a mérési eredmények az echokardiográfiás dokumentumokból anélkül, hogy közvetlen keresést végeznének, vagy részletes információval rendelkeznénk a dokumentum felépítéséről és az adatrögzítési szokásokról. A javasolt módszertan hatékonyan kezeli a helyesírási hibákat, a rövidítéseket és a leírásokban használt változatos terminológiát.