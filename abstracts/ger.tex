\centerline{\Large{\textbf{Abstrakt}}}

\vspace{0.5cm}

\noindent In meiner Dissertation geht es um die Nutzung von Gesundheitsdatenbeständen, in der ich mich auf zwei verschiedene Themen konzentriert habe: die Auswahl von Kontrollgruppen und die Extraktion von Informationen aus medizinischen Dokumenten. 

Beobachtungsstudien basieren häufig auf Fall-Kontroll-Studien. Bei solchen Studien wird die Schlussfolgerung (z. B. die Wirkung eines Arzneimittels auf die Heilung) auf der Grundlage des Vergleichs der Fallgruppe (behandelt) und der Kontrollgruppe (unbehandelt) gezogen. Das grundlegende Kriterium für die ordnungsgemäße Durchführung von Fall-Kontroll-Studien ist die Auswahl geeigneter Fall- und Kontrollgruppen. Bei retrospektiven Studien ist die behandelte Gruppe in der Regel vorgegeben und es muss für die Studie eine Kontrollgruppe ausgewählt werden, bei der die Individuen den Probanden der Fallgruppe hinsichtlich ihrer grundlegenden Merkmale, die die untersuchte Fragestellung beeinflussen, sehr ähnlich sind. In meiner Dissertation habe ich zwei neue, auf dem nächsten Nachbarn basierende Kontrollgruppenauswahlmethoden vorgeschlagen, die die Auswahl von Individuen im ursprünglichen n-dimensionalen Merkmalsraum durchführen. Ich habe die Wirksamkeit der vorgeschlagenen Methoden mit Monte-Carlo-Simulatio-nen unter Verwendung selbst vorgeschlagener Unähnlichkeitsmaße und anderer weit verbreiteter Ähnlichkeitsmaße getestet. 

Im medizinischen Alltag werden die Ergebnisse von Echokardiogrammen meist in Form von unstrukturiertem Text aufgezeichnet, aus dem die Extraktion relevanter Informationen eine anspruchsvolle Aufgabe darstellt. Um diese Informationsextraktion zu unterstützen, habe ich eine Text-Mining-basierte Informationsextraktionsmethode entwickelt, die die Beschreibungen der Herzultraschallmessung in den Befunden automatisch identifiziert, standardisiert und anschließend die extrahierten und standardisierten Messbeschreibungen zusammen mit den Messergebnissen in strukturierter Form speichert. Durch Fallstudien, die auf großen Datensätzen basieren, habe ich gezeigt, dass die vorgeschlagene Methode verwendet werden kann, um Messergebnisse aus echokardiographischen Dokumenten mit hoher Zuverlässigkeit zu extrahieren, ohne eine direkte Suche durchzuführen oder detaillierte Informationen über die Struktur des Dokuments und Datenaufzeichnungsgewohnheiten zu haben. Die vorgeschlagene Methodik behebt effektiv Rechtschreib-fehler, Abkürzungen und verschiedene in Beschreibungen verwendete Terminologien.
